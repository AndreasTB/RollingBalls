\sffamily
\chapter{Einf�hrung}
Im Fach Fortgeschrittene Spieleentwicklung soll von den Projektgruppen ein Spiel in C++ und OpenGL erstellt werden. Hierbei soll der Fokus vor allem auf den grafischen Effekten liegen.

Unsere Projektgruppe hat sich f�r einen Ableger des bekannten Spiels Tower Defense entschieden und das Projekt Rolling Balls genannt.

\section{Spielprinzip}
Rolling Balls besteht aus einer einfachen Fl�che als Spielumgebung. Ziel des Spiels ist es, rollende Kugeln zu zerst�ren, bevor diese das Ende der Spielfl�che erreichen. Die Spielfl�che wird in ein Grid unterteilt und der Spieler kann auf jedes Grid einen Turm bauen, der die heranrollenden Kugeln zerst�ren kann. In der ersten Implementierung werden die Kugeln jeden Turm, mit dem sie kollidieren besch�digen und zerst�ren, sofern sie nicht schnell von den T�rmen bek�mpft werden. 

\subsection{Schwierigkeitsgrad}
Der Spieler kann den Schwierigkeitsgrad selber w�hlen, hierbei stehen verschiedene M�glichkeiten zur Schwierigkeitssteuerung zur Verf�gung:
\begin{itemize}
	\item Rollgeschwindigkeit der Kugeln (je schneller, desto schwerer)
	\item Breite des Spielfelds (je breiter, desto schwerer)
	\item L�nge des Spielfelds (je l�nger, desto leichter)
\end{itemize}

\subsection{Economy}
Der Spieler braucht eine Menge X an Spielw�hrung, die er ben�tigt, um sich einen Turm zu kaufen.
F�r jede abgeschossene Kugel erh�lt der Spieler eine Summe an Geldw�hrung. Sobald der Spieler gen�gend Kugeln zerst�rt hat, kann er sich einen Turm kaufen.

\subsection{Spielende}
Es gibt kein fixes Spielende, jede Runde werden mehr Kugeln erzeugt, die aufgehalten werden m�ssen. Der Spieler hat eine bestimmte Anzahl an Leben und f�r jede Kugel, die das Ende des Spielfelds erreicht, wird dem Spieler ein Leben abgezogen. Sobald der Spieler keine Leben mehr hat, wird das Spiel beendet. Der Spieler erh�lt pro zerst�rter Kugel Spielpunkte, welche am Ende als Highscore angezeigt werden. Um Exploits w�hrend des Spiels zu minimieren, bekommt jede Kugel einen Countdown-Timer. Sobald der Countdown-Timer abgelaufen ist, explodiert die Kugel und zerst�rt im Umkreis von 2 Feldern in alle Richtungen die T�rme. Der Countdown-Timer wird f�r jede Zeile in Y-Richtung zur�ckgesetzt, d.h. man hat pro Y-Zeile eine gewisse Zeit, um die Kugel durchzuschleusen oder zu zerst�ren. Der Timer wird eingef�hrt, damit eine Kugel sich nicht eine endlos lange Zeit in ein und derselben Y-Zeile befindet.

\chapter{Prototyp}
Der Prototyp wird erstellt, um die generelle Spielidee vorzuf�hren. Hierbei wird auf THREE.js und WebGL zur�ckgegriffen.
Im Prototyp werden folgende Funktionalit�ten verf�gbar sein:
\begin{itemize}
	\item Darstellung des Spielfelds
	\item Einteilung des Spielfelds in Grids
	\item Erscheinen der Spielkugeln als THREE.Sphere
	\item T�rme werden als THREE.Cube dargestellt
	\item Interaktion mit dem Spielfeld zum Bauen der T�rme
\end{itemize}

was brauchen wir f�r den Prototypen
wer macht was
welche abh�ngigkeiten (was muss vor etwas anderem erledigt sein)
Projektplan